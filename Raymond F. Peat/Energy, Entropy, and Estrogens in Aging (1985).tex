\documentclass[10pt]{memoir}

\usepackage{fontspec}
\usepackage{microtype}
\usepackage{multicol}
\usepackage{fancyhdr}
\usepackage{hyperref}

\setstocksize{12.25in}{8in}
\settrimmedsize{12.25in}{8in}{*}
\settypeblocksize{11.25in}{6.5in}{*}
\setlrmarginsandblock{0.75in}{0.75in}{*}
\setulmarginsandblock{0.5in}{0.5in}{*}
\checkandfixthelayout

\hypersetup{
    hidelinks,
    pdftitle={Energy, Entropy, and Estrogens in Aging},
    pdfauthor={Raymond F. Peat}
}

\setmainfont{Pragmatica Web Book}

\newfontfamily\pragmatica{Pragmatica Web Bold}

\setlength{\topskip}{0mm}

\setcounter{page}{48}

\renewcommand{\headrulewidth}{0pt}

\makepagestyle{myfancypage}

\makeoddhead{myfancypage}{%
  \vbox to 0pt{%
    \vskip 0.5\headheight
    \vskip \headsep
    \vskip 0.5\textheight
    \hbox to \textwidth{%
      \hskip \textwidth
      \hskip \marginparsep
      {\pragmatica\Huge\thepage}
      \hss
    }%
  \vss
  }%
}{}{}

\makeevenhead{myfancypage}{}{}{%
  \vbox to 0pt{%
    \vskip 0.5\headheight
    \vskip \headsep
    \vskip 0.5\textheight
    \hbox to \textwidth{%
      \hss
      {\pragmatica\Huge\thepage}
      \hskip \textwidth
      \hskip \marginparsep
    }%
  \vss
  }%
}

\makeoddfoot{myfancypage}{}{}{}
\makeevenfoot{myfancypage}{}{}{}

\pagestyle{myfancypage}


\setlength{\columnsep}{30pt}

\begin{document}

\noindent\textit{Raymond Peat, PhD}

\noindent\textit{3977 Dillard Road}

\noindent\textit{Eugene, Oregon 97405 USA}

\medskip

\begin{center}
	{
		\Large%
		{\pragmatica ENERGY, ENTROPY, AND ESTROGENS IN AGING}
	}

	\bigskip

	\parbox{0.9\textwidth}{%
		\textit{Abstract: A physiological, biochemical and ecological concept of aging providing a basis\linebreak
			for therapy and prevention using anti-estrogen, anti-prolactin, and anti-calcium substances,\linebreak
			and a proposal for further research to investigate the mechanisms by which photo-period\linebreak
			modifies the biological processes and materials which control the rate of aging.}
	}
\end{center}

\medskip

\begin{multicols}{2}

	\noindent In my study of aging-changes in oxidative\linebreak
	metabolism, around 1970 I began to notice\linebreak
	parallels between aging and the effects of\linebreak
	estrogens. Investigating further, it appeared\linebreak
	that an energy deficit was the common fac-\linebreak
	tor, and that many harmful factors, including\linebreak
	radiation and hypoxia, caused a similar pat-\linebreak
	tern of changes. (Reference 1.)

	\bigskip

	\noindent A variety of biophysical methods (NMR and\linebreak
	ESR of living tissue, whole-cell electrophore-\linebreak
	sis, the influence of electromagnetic fields\linebreak
	on nerve cell latent periods, etc.) led me to\linebreak
	interpret the biochemical changes occurring\linebreak
	in various tissues in aging, hypoxia, and estro-\linebreak
	gen dominance as a coherent response of the\linebreak
	cells to disorder resulting from an excess ex-\linebreak
	penditure of energy. (References 1, 2 and 3.)

	\bigskip

	\noindent In 1971 I suggested that progesterone and\linebreak
	related steroids would have anti-aging proper-\linebreak
	ties deriving from their effects on the struc-\linebreak
	ture of cell water and the energy charge of\linebreak
	cell. (References 3 and 4.)

	\bigskip

	\noindent The same pattern of estrogen dominance is\linebreak
	seen in both sexes with aging, and in all\linebreak
	species studied. The anti-estrogens, especial-\linebreak
	ly progesterone and dehydroepiandrosterone\linebreak
	(DHEA) decline similarly with age in both\linebreak
	sexes.

	\bigskip

	\noindent Since 1971, I have worked out the physiolo-\linebreak
	gical implications of this approach, as I will\linebreak
	describe briefly below, and I have tested the\linebreak
	ideas in a wide variety of age-related condi-\linebreak
	tions in humans, including (1) Alzheimer's\linebreak
	disease and Parkinson's disease and other\linebreak
	«senile» changes in nerve function, (2) a\linebreak
	variety of senile degenerative skin changes\linebreak
	(3) changes in the circulatory system inclu-\linebreak
	ding arrhythmia, hypertension, gangrene of\linebreak
	the feet and putative «stroke,» (4) changes\linebreak
	in bones including osteoporosis and arthritis,\linebreak
	and (5) changes in respiratory function, in-\linebreak
	cluding emphysema-like changes appearing in\linebreak
	old age.

	\bigskip

	\noindent Progesterone and other «anti-estrogens» ha-\linebreak
	ve caused rapid and thorough recovery from\linebreak
	many of the diseases which are widely con-\linebreak
	sidered to be the unavoidable consequences\linebreak
	of aging. «Auto-immune» diseases have also\linebreak
	been treated successfully, regardless of age.\linebreak
	(References 5 and 6.)

	\bigskip

	\noindent {Many of the metabolic consequences of\parfillskip=0pt\par}

	\columnbreak

	\noindent stress tend to create vicious circles of self-\linebreak
	stimulating processes of decline, on both\linebreak
	the cellular and the systemic levels. The work\linebreak
	of F.Z. Meyerson, especially his concept of\linebreak
	the «calcium triad,» contributes powerful sup-\linebreak
	port to this view.

	\bigskip

	\noindent On the cellular level, in aging and estrogen\linebreak
	-dominance, Magnesium-ATP tends to be lost\linebreak
	as calcium is retained in excess. The sodium-\linebreak
	potassium-ATPase activity declines. Proteo-\linebreak
	lysis increasingly dominates relative to pro-\linebreak
	tein synthesis. Repolarization of nerve, mus-\linebreak
	cle, and secretory cell tends to be retarded,\linebreak
	causing such diverse events as «elevated\linebreak
	hypothalamus thresholds,» coronary vaso-\linebreak
	spasm, insomnia, hallucinations, delayed T-\linebreak
	wave in the electrocardiogram, and leg\linebreak
	cramps. Ammonia tends to appear in tissues\linebreak
	in these states, as protein catabolism domi-\linebreak
	nates.

	\bigskip

	\noindent On the systemic level in aging, estrogenic\linebreak
	effects are evident: a tendency toward pro-\linebreak
	lactinoma, toxic effects of prolactin on kid-\linebreak
	neys and other organs affecting water balan-\linebreak
	ce, disturbance of thyroid function, adrenal\linebreak
	compensation or exhaustion, atrophy of mus-\linebreak
	cle, skin, and other tissues. Prostate hyper-\linebreak
	trophy now appears to involve estrogen do-\linebreak
	minance, and to be reversible with progeste-\linebreak
	rone treatment. Reduced sensitivity to CO$_{2}$\linebreak
	in the blood, and diminished gas diffusion in\linebreak
	the lung also occur very frequently. Bowel\linebreak
	function is altered by prolaction toxicity, and\linebreak
	by other degenerative changes.

	\bigskip

	\noindent Since certain environmental events relate\linebreak
	clearly to the patterned metabolic changes\linebreak
	mentioned above, those events must be con-\linebreak
	sidered.

	\bigskip

	\noindent Animals of very different types migrate to\linebreak
	higher latitudes for reproduction. At high\linebreak
	latitudes the effect of the vernal increase\linebreak
	of photoperiod becomes intensified. The light-\linebreak
	induced hormonal changes which increase\linebreak
	fertility also increase general vitality and\linebreak
	decrease the problems associated with aging.\linebreak
	The hormonal changes occurring in winter\linebreak
	and at night aggravate degenerative proces-\linebreak
	ses and increase mortality in all of the spe-\linebreak
	cies that have been studied, including hu-\linebreak
	mans, in spite of the use of artificial lights.\linebreak
	Certain age-related conditions, especially os-\linebreak
	{teoporosis and depression, can be alleviated\parfillskip=0pt\par}

\end{multicols}

\newpage

\fancyhead[R]{%
	\vbox to0pt{%
			\vskip0.5\headheight
			\vskip\headsep
			\vskip0.5\textheight
			\hbox to\textwidth{%
				\hskip\textwidth
				\hskip\marginparsep
				{\pragmatica\Huge\thepage}
				\hss
			}%
			\vss
		}%
}
\fancyhead[L]{}

\begin{multicols}{2}

	\noindent with artificial illumination. It has been esta-\linebreak
	blished that darkness is associated with di-\linebreak
	minished anti-estrogens, and with elevation\linebreak
	of prolactin and cortisone. The normal diur-\linebreak
	nal cycle appears to be a factor in aging.

	\bigskip

	\noindent Loss of muscular tone in the bowel is asso-\linebreak
	ciated with aging, and increases the likelihood\linebreak
	of exposure to bacterial endotoxin and to\linebreak
	bacterial anti-thyroid substances, and increa-\linebreak
	ses the antigen burden. Permeability of the\linebreak
	bowel wall is increased by stress, and forms\linebreak
	part of another vicious circle, promoting de-\linebreak
	generative changes. These degenerative chan-\linebreak
	ges can be reversed partly by treatment with\linebreak
	hormones, but the altered bowel flora is a\linebreak
	problem which requires further research.

	\bigskip

	\noindent Treatment with progesterone and other anti-\linebreak
	estrogens has reversed a variety of age-re-\linebreak
	lated changes in humans and in experimental\linebreak
	animals, and has extended the life-span of\linebreak
	experimental animals.

	\bigskip

	\noindent A useful theory of aging should include three\linebreak
	levels: cellular chemistry, physiology, and\linebreak
	the relation of the organism to its environ-\linebreak
	ments. All of these levels can be involved\linebreak
	in self-stimulating processes, stabilizing the\linebreak
	organism at different levels of energy and\linebreak
	entropy in such a way as to make degenerative\linebreak
	{changes appear to be irreversible. Neverthe-\parfillskip=0pt\par}

	\columnbreak

	\noindent less, intervention at certain points to reverse\linebreak
	disorder is possible.

	\bigskip

	\noindent PROJECTED RESEARCH: I am investigating\linebreak
	the biochemistry, and potential therapeutic\linebreak
	use, of substances which seem to directly\linebreak
	or indirectly promote cellular order, repolari-\linebreak
	zation, and sensitivity to biological signals.\linebreak
	To further rationalize chemical therapy for\linebreak
	aging and its complications, I will try to de-\linebreak
	termine the mechanisms by which extended\linebreak
	photoperiods promote the desirable hormo-\linebreak
	nes. This will involve (1) examination of\linebreak
	the effects of light on DHEA levels (2), the\linebreak
	effects of photoperiod on intestinal flora, and\linebreak
	(3) investigation of physical-chemical chan-\linebreak
	ges in light sensitive tissues during stimula-\linebreak
	tion.

	\bigskip

	\noindent {\pragmatica REFERENCES:}

	\smallskip

	{%
		\small%
		\noindent 1. Age-related oxidative changes in the hamster ute-\linebreak
		rus, Ph.D. dissertation, University of Oregon, 1972.

		\noindent 2. Cold-Inactivated enzymes as metabolic controls,\linebreak
		Physiological Chemistry and Physics 3(4), 401, 1971.

		\noindent 3. Estrogen stimulated pathway changes and cold-\linebreak
		inactivated enzymes, Physiological Chemistry and\linebreak
		Physics 4(3), 295, 1972.

		\noindent 4. A biophysical approach to altered consciousness,\linebreak
		Journal of Orthomolecular Psychiatry, September,\linebreak
		1975.

		\noindent 5. Progesterone In Orthomolecular Medicine, 1979.

		\noindent 6. U.S. Patent 4, 439,432, new uses for a new com-\linebreak
		position of progesterone, March 27, 1984.%
	}

\end{multicols}

\noindent\rule{\textwidth}{0.4pt}\par

\bigskip

\noindent\textit{JOHANNA BAETZEL FOUNDATION FOR MOLECULAR BIOLOGY}

\noindent\textit{5 Terrace Street, Wilkes Barre, PA 18702, USA}

\bigskip

\begin{center}
	{\pragmatica\Large%
		COMPARISON OF THE BIOAVAILABILITY OF

		NUTRITIONAL NEUROTRANSMITTER

		PRE-CURSORS TO THE BRAIN, ALONE AND IN

		COMBINATION WITH DIFFERENT PENETRATION

		ENHANCEMENT FACTORS OF THE

		BLOOD-BRAIN-BARRIER (PILOT SELF STUDY)%
	}

	\bigskip

	\textit{Baruch Bezalel, M.D. Research Physician}
\end{center}

\begin{multicols}{2}

	\noindent INTRODUCTION

	\noindent A full function of the human brain depends\linebreak
	among others upon it's uninhibited ability of\linebreak
	it's predetermined cells to produce the neu-\linebreak
	rotransmitters acetylcholine and dopamine.\linebreak
	This ability of the brain cells is diminished\linebreak
	in degenerative brain processes (1,2,3,5),\linebreak
	which on the other hand can be restored by\linebreak
	the crucial biochemical compounds acetyl-\linebreak
	cholyne, dopamine and gangliosides (1,2,3,5,\linebreak
	6), whereby the latter are also called bio-\linebreak
	transmitter of membrane mediated informa-\linebreak
	tion (9).

	\noindent {To approach this full brain function in a 56\parfillskip=0pt\par}

	\columnbreak

	\noindent years old male individual (myself), nutritional\linebreak
	neurotransmitter precursors alone and in com-\linebreak
	bination with either blood brain barrier pe-\linebreak
	netrating factors such as dimethylamino-\linebreak
	ethanol (DMAE), (11) of thymus glandiosides\linebreak
	(8,9) were experimentally ingested and pos-\linebreak
	sible effects reported.

	\bigskip

	\noindent\textit{Materials and Methods}

	\bigskip

	\noindent The nutritional neurotransmitter precursors\linebreak
	were ingested at 5 pm every day by the\linebreak
	author.

	\bigskip

	\noindent Materials used were: 1-tyrosine (General\linebreak
	{Nutrition Corp.), choline (Solgay Co., Inc.),\parfillskip=0pt\par}

\end{multicols}

\end{document}